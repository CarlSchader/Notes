\documentclass{article}
\title{Packages}
\date{1/23/2020}
\author{Carl Schader}

\usepackage{amsmath}

\begin{document}
	\maketitle
	\pagenumbering{gobble}

	\section{Inline Math} % Use $equation$ to do inline equations.
	The formula $F = ma$ is an inline equation.

	\section{Equation and Align Environments}
	This equation uses the equation environment:
	\begin{equation*}
		y = mx + b
	\end{equation*}
	These equations uses the align environment:
	\begin{align*} % align will align equations at the &s automatically. The * is for the amsmath package.
		3x - 2x &= 15\\ % Seperate equations must be seperated by \\ (linebreak).
		x &= 15
	\end{align*}

	\section{Common Math Features}
	\begin{align*}
		f(x) &= x^2\\
		g(x) &= \frac{1}{x}\\
		F(x) &= \int^a_b \frac{1}{3}x^3\\
		h(x) &= \frac{1}{\sqrt{x}}\\
		h(x) &= \left(\frac{1}{\sqrt{x}}\right) %\left and \right make paranthesis or brackets scale with the objects inside of them.
	\end{align*}

	\section{Matrices} % matrix environment must be nested inside a math environment.
	\begin{align*}
		\begin{matrix} % &s in align and equation envs make matrices space correctly. White space (as far as I can tell) don't matter here.
			1&0&0\\
			0&1&0\\
			0&0&1
		\end{matrix}
	\end{align*}

	\begin{equation*}
		\left[
		\begin{matrix}
				1 & 0 & 0\\
				0 & 1 & 0\\
				0 & 0 & 1
		\end{matrix}
		\right]
	\end{equation*}
\end{document}
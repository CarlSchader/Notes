\documentclass{article}
\title{Types of Colonial Rule (Khapoya, p.127)}
\author{Carl Schader}
\date{January 28, 2020}

\begin{document}
\maketitle
% \tableofcontents

\section{Indirect (Brittish)}
	\begin{itemize}
	\item The Brittish were very clear that Africans could never be Brittish unlike the French.
	\item Very few Brittish civilians to look over their colonies.
	\item As long as the Brittish recieved their resources they didn't interfere with existing ethnic groups and power structures.
	\end{itemize}
	
\section{Direct (French)}
	\begin{itemize}
	\item France wanted to bring their colonies into the larger French culture as long as their subjects took part in their culture.
	\item Unlike the Brittish, the French would intervene in African power structures to complement French culture.
	\item The Brittish arguably had a better track record than the French in terms of development because the Brittish left cultures in tact allowing for easier development. The Brittish were also more focused on infrastructure and education. The Brittish were not trying to create black Brittain like France.
	\item Portugal and Italy also had a system similar to this.
	\end{itemize}
	
\section{Company (Belgian)}
	\begin{itemize}
	\item Belgian Congo.
	\item Probably the most brutal of all the models, the Belgian king gave the corporate powers in charge free rein to exploit the Congo without any regulation.
	\item Business interests took over completely.
	\end{itemize}

\section{Indirect Company (Brittish Variant)}
	\begin{itemize}
		\item Rhodesia is an example.
		\item These colonies were a buffer around South Africa to defend it from other colonizers.
		\item These colonies were called protectorates and were "protected" from other colonizers.
		\item The Brittish had even less interference in these colonies as their indirect ones. This is due to them being inland and more costly to govern.
	\end{itemize}

\end{document}